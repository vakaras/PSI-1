\newglossaryentry{siekis}{
  name=siekis,
  description={
    Siekiu vadinamas pagrindinis tikslas, kurio numatoma siekti naudojant 
    sukurtąją sistemą. Siekiu apibūdinama nauda, kurios galima tikėtis iš 
    sistemos, ir pagrindžiamas tos sistemos kūrimo tikslingumas
    %\cite[100]{cap_psi1} FIXME Išsiaiškinti kodėl neveikia.
    },
  sort=siekis,
  plural=siekiai
  }

\newglossaryentry{strateginis-tikslas}{
  name={strateginis tikslas},
  description={
    Strateginiai sistemos naudojimo tikslai – tai pagrindinio jos naudojimo
    tikslo potiksliai, nusakantys pačius bendriausius, ilgalaikius
    kuriamos sistemos užsakovo ketinimus
    %\cite[100]{cap_psi1} FIXME Išsiaiškinti kodėl neveikia.
    },
  sort={strateginis tikslas},
  plural={strateginiai tikslai}
  }

\newglossaryentry{operacinis-tikslas}{
  name={operacinis tikslas},
  description={
    Operaciniai sistemos naudojimo tikslai (sistemos sprendžiami 
    uždaviniai) – tai konstruktyvūs, matuojami ir terminuoti strateginių
    tikslų potiksliai
    %\cite[100]{cap_psi1} FIXME Išsiaiškinti kodėl neveikia.
    },
  sort={operacinis tikslas},
  plural={operaciniai tikslai}
  }

\newglossaryentry{duom_baz}{
  name={duomenų bazė},
  description={
    Duomenų rinkinys, susistemintas ir sutvarkytas taip, kad juo būtų
    galima patogiai naudotis
    %\cite{žod_komp} FIXME Išsiaiškinti kodėl neveikia.
    },
  sort={duomenų bazė},
  plural={duomenų bazės}
  }

\newglossaryentry{dok_sistema}{
  name={automatinė dokumentų kūrimo sistema},
  description={
    Sistema galinti pagal nurodytas taisykles sukurti tekstinį dokumentą
    ar laišką
                                        % FIXME Nėra klaidos?
    },
  sort={automatinė duomenų kūrimo sistema},
  plural={automatinės duomenų kūrimo sistemos}
  }

\newglossaryentry{p_serveris}{
  name={pašto serveris},
  description={
    Serveris, naudojamas elektroniniams laiškams persiųsti kompiuterių
    tinklais ir juos pristatyti gavėjams.
    %\cite{žod_komp} FIXME Išsiaiškinti kodėl neveikia.
    },
  sort={pašto serveris},
  plural={pašto serveriai}
  }

\newglossaryentry{tinklalapis}{
  name=tinklalapis,
  description={
    Informacijos išteklius saityne, kuris gali būti pasiektas naudojantis
    naršykle
    %\cite{žod_komp} FIXME Išsiaiškinti kodėl neveikia.
    },
  sort=tinklalapis,
  plural=tinklalapiai,
  user1=tinklalapių
  }
  \newcommand{\glstinklalapisdgsk}{\glsuseri{tinklalapis}}

\newglossaryentry{svetaine}{
  name=svetainė,
  description={
    Rinkinys \glstinklalapisdgsk, kuriuos sieja bendra tematika, 
    priklausomybė vienai įstaigai ar kitokie bendri dalykai
    %\cite{žod_komp} FIXME Išsiaiškinti kodėl neveikia.
    },
  sort=svetainė,
  plural=svetainės
  }

\newglossaryentry{darb_kl}{
  name={darbo klaida},
  description={
    Tiksliai apibrėžiamas neatitikimas tarp gautojo darbo rezultato ir
    to, kurio buvo tikėtasi. Pavyzdžiui: rėmėjo pavadinimas atspausdintas
    su klaida arba kvietimas išsiųstas ne tam mokiniui, kuriam reikėjo
    },
  user1={darbo klaidų},
  sort={darbo klaida},
  plural={darbo klaidos}
  }
  \newcommand{\glsdarbkldgsk}{\glsuseri{darb_kl}}

\newglossaryentry{saitynas}{
  name={saitynas},
  description={
    Hipertekstinės informacijos visuotinis tinklas, svarbiausias interneto
    komponentas. (Angliškai: \emph{Web 2.0})
    },
  user1={saitynu},
  sort={saitynas},
  plural={saitynai}
  }
  \newcommand{\glssaitynasvnsi}{\glsuseri{saitynas}}

\newglossaryentry{spec_poreik}{
  name={specialusis poreikis},
  description={
    Poreikiai, norai, pageidavimai, kurių turi tik vienas mokinys. 
    Pavyzdžiui, noras dalyvauti tik pusėje stovyklos, nes stovyklos vykimo
    data kertasi su olimpiados, kurioje mokinys dalyvauja, data
    },
  user1={specialiuosius poreikius},
  sort={specialusis poreikis},
  plural={specialieji poreikiai}
  }
  \newcommand{\glsspecporeikdgsg}{\glsuseri{spec_poreik}}

\newglossaryentry{darb_poz}{
  name={darbo požymiai},
  description={
    Dominančios informacijos apie konkretų darbą rinkinys. Įskaitant, bet
    neapsiribojant: darbo pavadinimas, užduoties aprašymas, vidutinė
    atlikimo trukmė, leidimai, kuriuos turi turėti \gls{vykdytojas}, 
    norintis imtis šio darbo
    },
  user1={darbo požymius},
  sort={darbo požymiai},
  plural={darbų požymiai}
  }
  \newcommand{\glsdarbpozvnsg}{\glsuseri{darb_poz}}

\newglossaryentry{vykdytojas}{
  name={vykdytojas},
  description={
    Žmogus atliekantis darbus
    },
  user1={vykdytojus},
  sort={vykdytojas},
  plural={vykdytojai}
  }
  \newcommand{\glsvykdytojasdgsg}{\glsuseri{vykdytojas}}

\newglossaryentry{virt_serv}{
  name={virtuali tarnybinė stotis},
  description={
    Virtualus tinklo mazgas, atliekantis tam tikras funkcijas pagal kitų
    tinkle esančių klientų kompiuterių paraiškas
    },
  sort={virtuali tarnybė stotis},
  plural={virtualios tarnybinės stotys}
  }
