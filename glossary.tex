\newglossaryentry{siekis}{
  name=siekis,
  description={
    Siekiu vadinamas pagrindinis tikslas, kurio numatoma siekti naudojant 
    sukurtąją sistemą. Siekiu apibūdinama nauda, kurios galima tikėtis iš 
    sistemos, ir pagrindžiamas tos sistemos kūrimo tikslingumas
    %\cite[100]{cap_psi1} FIXME Išsiaiškinti kodėl neveikia.
    },
  sort=siekis,
  plural=siekiai
  }

\newglossaryentry{strateginis-tikslas}{
  name={strateginis tikslas},
  description={
    Strateginiai sistemos naudojimo tikslai – tai pagrindinio jos naudojimo
    tikslo potiksliai, nusakantys pačius bendriausius, ilgalaikius
    kuriamos sistemos užsakovo ketinimus
    %\cite[100]{cap_psi1} FIXME Išsiaiškinti kodėl neveikia.
    },
  sort={strateginis tikslas},
  plural={strateginiai tikslai}
  }

\newglossaryentry{operacinis-tikslas}{
  name={operacinis tikslas},
  description={
    Operaciniai sistemos naudojimo tikslai (sistemos sprendžiami 
    uždaviniai) – tai konstruktyvūs, matuojami ir terminuoti strateginių
    tikslų potiksliai
    %\cite[100]{cap_psi1} FIXME Išsiaiškinti kodėl neveikia.
    },
  sort={operacinis tikslas},
  plural={operaciniai tikslai}
  }

\newglossaryentry{saitynas}{
  name={saitynas},
  description={
    Hipertekstinės informacijos visuotinis tinklas, svarbiausias interneto
    komponentas
    },
  sort=saitynas
  }

\newglossaryentry{duomenų-bazė}{
  name={duomenų bazė},
  description={
    Duomenų rinkinys, susistemintas ir sutvarkytas taip, kad juo būtų
    galima patogiai naudotis
    },
  sort={duomenų bazė},
  plural={duomenų bazės}
  }
