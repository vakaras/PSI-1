\chapter{Užsakovo poreikių analizė}

% TODO Suvesti terminus į žodynėlį.

\ref{tab:poreikiai} lentelėje pateiktas sąrašas kompiuterinių resursų, 
kurių reikia norinti įgyvendinti \emph{\nameref{section_strat_oper_tiksl}}
skyrelyje aprašytiems operaciniams tikslams.

\begin{table}
  \centering
  \begin{tabular}[]{| l | p{1.6cm} | p{5.8cm} | c |}
    \hline
    Nr. & Operacinis tikslas & Resursai, reikalingi tikslui įgyvendinti &
    Prioritetas \\
    \hline
    1. & \ref{tiksl_el} & 
      \Gls{pašto-serveris}; \gls{duomenų-bazė} su moksleivių duomenimis; 
      \gls{automatinė-dokumentų-kūrimo-sistema}. & 1 \\
    \hline
    2. & \ref{tiksl_r} & 
      \Gls{svetainė}, jos valdymo ir priežiūros įrankiai; 
      \gls{duomenų-bazė} moksleivių duomenims saugoti; 
      \gls{automatinė-dokumentų-kūrimo-sistema}. & 1 \\
    \hline
    3. & \ref{tiksl_dvisk} &
      \Gls{duomenų-bazė} duomenims apie darbus saugoti. & 2 \\
    \hline
    4. & \ref{tiksl_dbus} &
      \Gls{duomenų-bazė} duomenims apie darbus saugoti. & 2 \\
    \hline
    5. & \ref{tiksl_dinfo} &
      \Gls{duomenų-bazė} duomenims apie darbus saugoti. & 2 \\
    \hline
    6. & \ref{tiksl_dvyk} &
      \Gls{duomenų-bazė} duomenims apie darbus ir jų vykdytojus 
      saugoti. & 3 \\
    \hline
    7. & \ref{tiksl_dv} &
      \Gls{duomenų-bazė} duomenims apie darbus, bei vykdytojus ir 
      jiems suteiktas teises, saugoti. & 3 \\
    \hline
  \end{tabular}
  \caption{Užsakovo poreikių analizės lentelė}
  \label{tab:poreikiai}
\end{table}
