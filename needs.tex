\chapter{Užsakovo poreikių analizė}

% TODO Suvesti terminus į žodynėlį.

\ref{tab:poreikiai} lentelėje pateiktas sąrašas kompiuterinių resursų, 
kurių reikia norinti įgyvendinti \emph{\nameref{section_strat_oper_tiksl}}
skyrelyje aprašytiems operaciniams tikslams.

Iš pateiktų operacinių tikslų, matome, jog siekiui įgyvendinti reikalinga
programų sistema galinti:
\begin{itemize}
  \item siųsti elektroninius laiškus;
  \item saugoti darbų sąrašus su informacija apie kiekvieną iš jų;
  \item priimti informaciją iš moksleivių.
\end{itemize}

\begin{table}
  \centering
  \begin{tabular}[]{| l | p{1.6cm} | p{5.8cm} | c |}
    \hline
    Nr. & Operacinis tikslas & Resursai, reikalingi tikslui įgyvendinti &
    Prioritetas \\
    \hline
    1. & \ref{tiksl_el} & 
      \Gls{p_serveris}; \gls{duom_baz} su moksleivių duomenimis. & 1 \\
    \hline
    2. & \ref{tiksl_r} & 
      \Gls{svetaine}, jos valdymo ir priežiūros įrankiai; 
      \gls{duom_baz} moksleivių duomenims saugoti. & 1 \\
    \hline
    3. & \ref{tiksl_dvisk} &
      \Gls{duom_baz} duomenims apie darbus saugoti. & 2 \\
    \hline
    4. & \ref{tiksl_dbus} &
      \Gls{duom_baz} duomenims apie darbus saugoti. & 2 \\
    \hline
    5. & \ref{tiksl_dinfo} &
      \Gls{duom_baz} duomenims apie darbus saugoti. & 2 \\
    \hline
    6. & \ref{tiksl_dvyk} &
      \Gls{duom_baz} duomenims apie darbus ir jų vykdytojus 
      saugoti. & 3 \\
    \hline
    7. & \ref{tiksl_dv} &
      \Gls{duom_baz} duomenims apie darbus, bei vykdytojus ir 
      jiems suteiktas teises, saugoti. & 3 \\
    \hline
    8. & \ref{tiksl_vad} &
      \Gls{duom_baz} duomenims apie darbus saugoti. & 2 \\
    \hline
  \end{tabular}
  \caption{Užsakovo poreikių analizės lentelė}
  \label{tab:poreikiai}
\end{table}

