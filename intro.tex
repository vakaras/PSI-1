\chapter{Įvadas}

% TODO: žr. 5 reikalavimų psl.

\section{Programų sistemos pavadinimas}
Pilnas programų sistemos pavadinimas – „Nuotolinio mokymo mokyklos padalinio, 
organizuojančio stovyklas pažangiems moksleiviams, pagalbinė darbų 
valdymo ir skirstymo sistema“.
Trumpas programų sistemos pavadinimas – „Skirstytuvas“.

\section{Dalykinė sritis}
Stovyklų pažangiems moksleiviams organizavimas.

\section{Probleminė sritis}
Efektyvus darbų pasiskirstymas ir mechaninių darbų kiekio sumažinimas.

\section{Naudotojai}
\begin{tabular}{|c|p{7cm}|}
  \hline 
  Naudotojas & Kvalifikacija \\
  \hline
  Vadovas & Kompiuterinio raštingumo pagrindai, sistemos „Skirtstytuvas“ 
  administravimo apmokymas \\
  \hline
  Vykdytojas & Kompiuterinio raštingumo pagrindai, sistemos „Skirtstytuvas“ 
  naudojimo apmokymas \\
  \hline
  Moksleivis & gebėjimas naudoti saitynu \\% TODO ideti nuoroda
  \hline
\end{tabular}

\section{Darbo pagrindas}
Dokumentas yra parengtas kaip programų sistemos inžinerijos laboratorinis 
darbas. Darbas parengtas remiantis % TODO nuoroda i reikalvimus
reikalavimais.

%TODO: ištrinti, kai bus normalių citatų.
Beprasmė citata iš Čaplinsko vadovėlio: „Objektinio stiliaus architektūros
grindžiamos objektine paradigma. Ji šiuo metu yra 
vyraujanti.“\cite[50]{cap_psi2}

