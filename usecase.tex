\chapter{Sistemos naudojimo scenarijus}

\section{Esamoji būklė}

Mokyklos būstinėje yra vienas nešiojamas kompiuteris su Windows operacine
sistema, kuriame laikomi visi „einamieji“ duomenys, bei prie jo prijungtas 
lazerinis spausdintuvas. Nuotolinio mokymo mokykla taip pat turi savo 
svetainę, bei duomenų bazę su informacija apie moksleivius ir dėstytojus.
Darbuotojai dirba su savo asmeniniais kompiuteriais, kuriuose naudoja
įvairią programinę įrangą (pavyzdžiui naudojamų operacinių sistemų 
sąraše yra „Windows XP“, „Windows 7“, „Ubuntu Linux“, „Xubuntu Linux“,
„Gentoo Linux“). Dauguma darbuotojų turi didelę darbo su biuro 
programomis patirtį.
% TODO Išversti ir papildyti.

\section{Scenarijaus aprašas}

\ref{fig:uml_usecase} diagramoje pavaizduota darbo su sistema UML schema.

\begin{figure}[htb]
  \begin{center}
    %\includegraphics[]{images/usecase.png}
  \end{center}
  \caption{UML schema vaizduojanti darbą su įdiegta sistema.}
  \label{fig:uml_usecase}
\end{figure}

\section{Priemonės scenarijui įgyvendinti}
