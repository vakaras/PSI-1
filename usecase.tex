\chapter{Sistemos naudojimo scenarijus}

\section{Esamoji būklė}

Mokyklos būstinėje yra vienas nešiojamas kompiuteris su Windows operacine
sistema, kuriame laikomi visi „einamieji“ duomenys, bei prie jo prijungtas 
lazerinis spausdintuvas. Nuotolinio mokymo mokykla taip pat turi savo 
svetainę, bei duomenų bazę su informacija apie moksleivius ir dėstytojus.
Darbuotojai dirba su savo asmeniniais kompiuteriais, kuriuose naudoja
įvairią programinę įrangą (pavyzdžiui naudojamų operacinių sistemų 
sąraše yra „Windows XP“, „Windows 7“, „Ubuntu Linux“, „Xubuntu Linux“,
„Gentoo Linux“). Dauguma darbuotojų turi didelę darbo su biuro 
programomis patirtį.

\section{Scenarijaus aprašas}

\ref{fig:uml_usecase} diagramoje pavaizduota darbo su sistema UML schema.
\hfill \\

Vadovas informuoja sistemą, kad stovyklos organizavimas yra pradedamas.
Jis į sistemą įkelia darbų sąrašą, kiekvienam darbui nurodydamas atributus.
Po to         % TODO : Atributus įkelti į žodyną
prideda vykdytojus, kurie atlikinės darbus iš darbų sąrašo. Vos pridėjus
vykdytoją, sistema jį apie tai informuoja elektroniniu laišku. 

Vykdytojas prisijungęs prie
sistemos gali peržiūrėti darbų sąrašą. Jis pasirenka
darbą, kurį atlikinės. Baigęs pasirinkto darbo vykdymą, pažymi tą
darbą kaip atliktą. Sistema informuoja vadovą apie darbo atlikimą ir
paprašo tai patvirtinti. Vadovui atmetus darbo atlikimą, sistema informuoja
apie tai vykdytoją prašydama pataisyti darbą. Vykdytojui patvirtinus
apie darbo pataisymą, sistema vėl prašo vadovo patvirtinti darbo kokybę. 
Taip daroma tol, kol vadovas patvirtina darbo kokybę. 

Vėliau vadovas prideda kviečiamų moksleivių sąrašą. Sistema jiems išsiunčia
kvietimus reikalaudama patvirtinti, jog dalyvaus, arba pranešti, kad negalės 
atvykti. Patvirtinusių dalyvavimą moksleivių paprašoma patikslinti jų 
asmeninius duomenis, patvirtinti jų teisingumą. Kai užbaigiamas tvarkaraščio
sudarymo darbas, sistema nusiunčia visiems dalyvausiantiems moksleiviams
tvarkaraščio kopiją.

Pasibaigus stovyklai vadovas praneša apie tai sistemai. Ji moksleiviams
išsiuntinėja stovyklos vertinimo anketas bei darbų vykdytojų paprašo 
pateikti stovyklos raportą. Sistema, gavusi visus grįžtamuosius atsakymus,
suformuoja ataskaitą ir ją pateikia stovyklos vadovui.

\begin{figure}[htb]
  \begin{center}
    \includegraphics[scale=0.7]{images/Seka.png}
    \caption{UML schema vaizduojanti darbą su įdiegta sistema.}
  \end{center}
  \label{fig:uml_usecase}
\end{figure}


\subsection{Darbo vietų aprašas}
\begin{enumerate}
  \item \emph{Vadovas}
	\begin{itemize}
	  \item Techninė įranga:
		\begin{enumerate}
			\item kompiuteris su internetu.
		\end{enumerate}
	  \item Programinė įranga:
		\begin{enumerate}
			\item operacinė sistema;
      \item interneto naršyklė.
		\end{enumerate}
	  \item Kvalifikaciniai reikalavimai:
		\begin{enumerate}
			\item kompiuterinio raštingumo pagrindai;
			\item sistemos „Skirstytuvas“ administravimo apmokymas.
		\end{enumerate}
	\end{itemize}

  \item \emph{Vykdytojas}
	\begin{itemize}
	  \item Techninė įranga:
		\begin{enumerate}
			\item kompiuteris su internetu.
		\end{enumerate}
	  \item Programinė įranga:
		\begin{enumerate}
			\item operacinė sistema;
      \item interneto naršyklė.
		\end{enumerate}
	  \item Kvalifikaciniai reikalavimai:
		\begin{enumerate}
			\item kompiuterinio raštingumo pagrindai;
			\item sistemos „Skirstytuvas“ naudojimo apmokymas.
		\end{enumerate}
	\end{itemize}
\end{enumerate}

\section{Priemonės scenarijui įgyvendinti}
\begin{enumerate}
	\item Virtuali tarnybinė stotis
	\item Tinklo įranga
	\item Pašto serveris
	\item Duomenų bazių valdymo sistemos
	\item Interneto ryšio paslaugos
	\item Operacinės sistemos
	\item Darbuotojų apmokymas naudotis sistema
	\item Interneto svetainės vardo sritis .lt zonoje
\end{enumerate}