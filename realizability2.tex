\section{Juridinis įgyvendinamumas}

Kuriama sistema neprieštarauja Lietuvos Respublikos įstatymams. Stovyklos dalyvių asmens duomenų rinkimas neprieštarauja Lietuvos Respublikos Asmens duomenų teisinės apsaugos įstatymui. Kiekvienam dalyviui bus suteikiama galimybė nesutikti su jo asmens duomenų rinkimu, arba davus leidimą peržiūrėti jau surinktus duomenis ir panorėjus juos pakeisti. Pasibaugus stovyklos organizavimo ir vykdymo laikotarpiui nereikalingi duomenys bus sunaikinami. Taip pat numatoma laikytis visų duomenų saugumo kriterijų.

\section{Sistemos panaudojimas}

% TODO

\ref{fig:uml_tasks} kažkas pavaizduota?

\begin{figure}[htb]
  \begin{center}
    \includegraphics[scale=0.8]{images/sistemos_panaudojimas.png}
  \end{center}
  \caption{UML schema vaizduojamos aktorių vykdomos užduotys, ir kaip jas
    padeda vykdyti sistema.}
  \label{fig:uml_tasks}
\end{figure}

\section{Sistemos teikiama nauda}

% TODO
\ref{fig:uml_tasks2} kažkas pavaizduota?

\begin{figure}[htb]
  \begin{center}
    \includegraphics[scale=0.8]{images/sistemos_tiekiama_nauda.png}
  \end{center}
  \caption{UML schema vaizduojamos aktorių vykdomos užduotys, ir kaip jas
    padeda vykdyti sistema.}
  \label{fig:uml_tasks2}
\end{figure}
