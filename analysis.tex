\chapter{Verslo proceso analizė}

\section{Verslo proceso aprašas}

Nuotolinio mokymo mokyklos padalinys, organizuojantis stovyklas pažangiems
mokiniams, yra atsakingas už vidutiniškai dešimties dienų trukmės
stovyklos vidutiniškai šimtui mokinių suorganizavimą. Šis padalinys
rūpinasi nurodytų kaip pažangių mokinių bei jų dėstytojų pakvietimu, 
apgyvendinimu bei maitinimu. Taip pat jis atsakingas už stovyklos metu 
skaitomų paskaitų kokybę bei mokinių aprūpinimą visomis reikalingomis
darbui priemonėmis. Didžiąją dalį stovyklos organizavimui reikalingų lėšų
duoda nuotolinio mokymo mokyklos rėmėjai, o trūkumą padengia į stovyklą
važiuojantys mokiniai, sumokėdami dalyvio mokestį. Su mokiniais ir
dėstytojais dažniausiai bendraujama individualiais elektroniniais 
laiškais, o duomenys saugomi popieriuje bei MS Excel skaičiuoklės knygose.

\section{Išorinė verslo proceso analizė}

\subsection{Analizės rezultatai}

Stovyklų tikslas yra suteikti mokiniams galimybę pagilinti
savo žinias ir kitaip patobulėti. Taigi, pagrindinė proceso įeiga ir išeiga
yra tie patys mokiniai. Kviečiant į stovyklą mokinius, yra
žinomi jų nuotolinio mokymosi rezultatai bei kokia jų dalis priima 
kvietimą dalyvauti stovykloje (įeigos kriterijai). Praėjus šiek tiek
laiko po stovyklos vėl galima įvertinti mokinių nuotolinio
mokymosi rezultatus. Be to, mokiniai užpildo įvertinimo anketas, kuriose
nurodo, ar jiems patiko konkrečios paskaitos ir visa stovykla
(išeigos kriterijai). Taip pat organizuojant stovyklą yra svarbi jos 
organizavimo kaina (nes jai didėjant dalyvio mokestis kai kuriems 
mokiniams gali tapti per didelis) bei organizuojant padarytų klaidų
(nusižengimai įstatymams, susitarimams bei \glspl{darb_kl}) skaičius. 

Apibendrinant, galima išskirti tokius stovyklų organizavimo proceso 
vertinimo kriterijus (jų kritinės vertės pateiktos \ref{tab:krit}
lentelėje):
\begin{enumerate}
  \item \label{krit_motyvacija}
    motyvacijos mokytis pokytis – kiek padidėjo
    ar sumažėjo mokinio teisingai išspręstų ir visų jam duotų užduočių
    santykis; 
  \item \label{krit_noras}
    mokinių noras dalyvauti stovykloje – kokia 
    dalis mokinių, gavusių kvietimą, pareiškia norą dalyvauti;
  \item \label{krit_pasitenkinimas}
    mokinių atsiliepimai apie stovyklą pasitenkinimo anketoje;
  \item \label{krit_sanaudos}
    stovyklos organizavimo sąnaudos;
  \item \label{krit_pazeidimai}
    juridinių aktų – įstatymų (buhalterinės apskaitos,
    mokesčių administravimo, asmens duomenų teisinės apsaugos
    ir kiti) bei susitarimų – pažeidimų skaičius;
  \item \label{krit_dklaidos}
    padarytų \glsdarbkldgsk {} skaičius;
  \item \label{krit_klaidos}
    neištaisytų klaidų skaičius;
  \item \label{krit_verte}
    pažeidimų  ir klaidų sukeltų padarinių vertė – kiek kainuotų juos 
    pašalinti, jei būtų bandoma tai daryti bei moralinė žala.

\end{enumerate}

\begin{table}[h!]
  \centering
  \begin{tabular}[]{| l | l | c |}
    \hline
    Kriterijus & Matas & Kritinė vertė \\
    \hline
    \ref{krit_motyvacija} & Procentai & < -5\% (sumažėjo 5\%) \\
    \hline
    \ref{krit_noras} & Procentai & < 80\% \\
    \hline
    \ref{krit_pasitenkinimas} & Dešimtbalė sistema & < 7 \\
    \hline
    \ref{krit_sanaudos} & Litais dalyviui & > 2000Lt \\
    \hline
    \ref{krit_pazeidimai} & Vienetais & > 0 \\
    \hline
    \ref{krit_dklaidos} & Vienetais & 
    $\infty$\footnotemark[1]\\
    \hline
    \ref{krit_klaidos} & Vienetais & > 0 \\ 
    \hline
    \ref{krit_verte} & Litais & > 1000Lt \\
    \hline
  \end{tabular}
  \caption{Kriterijų kritinės vertės}
  \label{tab:krit}
\end{table}

\footnotetext[1]{Neribojama. Svarbu ne padarytų, o neištaisytų 
      klaidų skaičius.}
\subsection{Problemos, grėsmės ir neišnaudotos galimybės}

Problemos:
\begin{itemize}
  \item Didelis paliktų neištaisytų klaidų skaičius.
\end{itemize}

Grėsmės:
\begin{itemize}
  \item Paramos, gaunamos iš rėmėjų mažėjimas. 
  \item Mokinių noro dalyvauti mažėjimas.
\end{itemize}

Neišnaudotos galimybės:
\begin{itemize}
  \item Nėra atsižvelgiama į galimus mokinių \glsspecporeikdgsg.
\end{itemize}

\section{Vidinė verslo proceso analizė}

Atlikus analizę paaiškėjo, jog probleminis klaidų skaičius atsiranda dėl
šių priežasčių:
\begin{enumerate}
  \item visi darbai daromi rankomis (pavyzdžiui: nors ir kvietimai
    siunčiami elektroniniu paštu, bet jų tekstas yra renkamas ranka);
  \item netikėti vėlavimai atliekant proceso grandinės dalis. Dėl jų 
    negalima atlikti vėlesnių tai grandinei priklausančių darbų, nes
    jie yra priklausomi nuo „užstrigusio“ darbo. Dėl to paskutinę 
    savaitę susikaupia darbų, su kuriais nebespėjama susitvarkyti.
\end{enumerate}

Ši problema taip pat daro įtaką grėsmių atsiradimui, nes:
\begin{enumerate}
  \item padidėjus paliktų klaidų skaičiui rėmėjai nusprendžia, kad 
    stovyklos organizatoriai nėra pajėgūs kokybiškai panaudoti jų paramą,
    ir todėl sumažina finansavimą. Dėl to padidėja dalyvio mokestis, ir
    dalis mokinių nebegali dalyvauti;
  \item kritus stovyklos organizavimo kokybei sumažėja stovyklos
    populiarumas.
\end{enumerate}

\section{Verslo tobulinimo strategija}

Pagrindinis verslo tobulinimo siekis yra padidinti darbo efektyvumą ir 
sumažinti klaidų skaičių.

\section{Strateginiai ir operaciniai tikslai verslo tobulinimo %
  strategijai įgyvendinti} \label{section_strat_oper_tiksl}

\begin{easylist}
& \label{tiksl_efek} 
  Padidinti darbo efektyvumą.
&&  \label{tiksl_auto} 
    Įdiegti priemones, leidžiančias automatizuoti mechaninius darbus:
&&& \label{tiksl_el} 
      kvietimų mokiniams siuntimas;
&&& \label{tiksl_r} 
      informacijos iš mokinių surinkimas.
& \label{tiksl_kl} 
  Sumažinti klaidų skaičių.
&& \label{tiksl_vald} 
    Įdiegti priemones, leidžiančias efektyviai sekti ir valdyti darbo eigą:
&&& \label{tiksl_dvisk} 
      vienoje vietoje matyti visų darbų sąrašą bei kaip jų vykdymas 
      atitinka numatytąjį planą;
&&& \label{tiksl_dbus} 
      peržiūrėti darbo būseną (negalimas pradėti vykdyti, nes nebaigti 
      kiti darbai, nuo kurių šis priklauso; galimas pradėti vykdyti; 
      pradėtas vykdyti; baigtas vykdyti; patikrintas; kada vėliausiai 
      turi būti pradėtas, kad būtų baigtas laiku; jei vėluoja, tai kiek);
&&& \label{tiksl_dinfo} 
      % TODO Patikrinti ar iš tikrųjų: 
      % „darbo informacija“ = „darbo požymiai“
      peržiūrėti \glsdarbpozvnsg {} (ką konkrečiai reikia atlikti; kokius 
      darbus reikia atlikti prieš tai; kiek vidutiniškai jį užtrunka 
      atlikti; kokios dažniausiai iškyla problemos jį vykdant);
&&& \label{tiksl_dvyk} 
      peržiūrėti \glsvykdytojasdgsg {} (kas vykdė/vykdo šį darbą; kas 
      atliko peržiūrą);
&&& \label{tiksl_dv} 
      kiekvienas \gls{vykdytojas} (jei jam suteiktos tokios teisės) 
      gali pasiimti darbą vykdymui;
&&& \label{tiksl_vad}
      vadovas gali patikrinti, ar gerai atliktas darbas, ir 
      patvirtinti/atmesti jo atlikimą.
\end{easylist}
