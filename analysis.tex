\chapter{Verslo proceso analizė}

\section{Verslo proceso aprašas}

Nuotolinio mokymo mokyklos padalinys, organizuojantis stovyklas pažangiems
moksleiviams, yra atsakingas už vidutiniškai dešimties dienų trukmės
stovyklos vidutiniškai šimtui moksleivių suorganizavimą. Šis padalinys
rūpinasi nurodytų, kaip pažangių, moksleivių, bei jų dėstytojų pakvietimu, 
apgyvendinimu, bei maitinimu. Taip pat jis atsakingas už stovyklos metu 
skaitomų paskaitų kokybę, bei moksleivių aprūpinimą visomis reikalingomis
darbui priemonėmis. Didžiąją dalį stovyklos organizavimui reikalingų lėšų
duoda nuotolinio mokymo mokyklos rėmėjai, o trūkumą padengia į stovyklą
važiuojantys moksleiviai, sumokėdami dalyvio mokestį. Su moksleiviais ir
dėstytojais dažniausiai bendraujama elektroniniais individualiais 
laiškais, o duomenys saugomi popieriuje, bei MS Excel skaičiatlentėse.

\section{Išorinė verslo proceso analizė}

\subsection{Analizės rezultatai}

Stovyklų organizavimo tikslas yra suteikti moksleiviams galimybę pagilinti
savo žinias ir kitaip patobulėti. Taigi pagrindinė proceso įeiga ir išeiga
yra tie patys moksleiviai. Apie į stovyklą kviečiamus moksleivius, mes
žinome, jų nuotolinio mokymosi rezultatus, bei kokia jų dalis priima 
kvietimą dalyvauti stovykloje (įeigos kriterijai). Po stovyklos
praėjus laiko tarpui, mes vėl galime įvertinti moksleivių nuotolinio
mokymosi rezultatus, be to moksleiviai užpildo įvertinimo anketas, kuriose
nurodo, kaip jiems patiko konkrečios paskaitos ir visa stovykla apskritai
(išeigos kriterijai). Taip pat organizuojant stovyklą yra svarbu jos 
organizavimo kaina (nes jai didėjant dalyvio mokestis kai kuriems 
moksleiviams gali tapti per dideliu), bei organizuojant padarytų klaidų
(nusižengimai įstatymams, susitarimams, bei darbo klaidos) kiekis.

Apibendrindami, galime išskirti tokius stovyklų organizavimo proceso 
vertinimo kriterijus:
\begin{enumerate}
  \item motyvacijos mokytis pokytis (motyvacija mokytis) – kiek padidėjo
    ar sumažėjo moksleivio teisingai išspręstų ir visų jam duotų užduočių
    santykis;
  \item moksleivių noras dalyvauti stovykloje (noras dalyvauti) – kokia 
    dalis moksleivių, gavusių kvietimą, pareiškia norą dalyvauti;
  \item moksleivių atsiliepimai apie stovyklą pasitenkinimo anketoje;
  \item stovyklos organizavimo sąnaudos (kaina);
  \item juridinių aktų (įstatymų (buhalterinės apskaitos,
    mokesčių administravimo, asmens duomenų teisinės apsaugos
    ir kiti), bei susitarimų) pažeidimų skaičius;
  \item padarytų darbinių klaidų (pavyzdžiui rėmėjo pavadinimas 
    atspausdintas su klaida, arba kvietimas išsiųstas ne tam moksleiviui, 
    kuriam reikėjo) skaičius;
  \item neištaisytų klaidų skaičius;
  \item pažeidimų  ir klaidų sukeltų padarinių vertė – kiek kainuotų juos 
    pašalinti, jei būtų bandoma tai daryti + moralinė žala;
  \item didžiausias galimas turimos komandos aptarnaujamų moksleivių 
    skaičius.
\end{enumerate}

Kriterijų detalizavimas, ir jų kritinės vertės:

\begin{tabular}[]{| l | p{2.2cm} | c |}
  \hline
  Kriterijus & Matas & Kritinė vertė \\
  \hline
  Motyvacija mokytis & Procentai & < -5\% (sumažėjo 5\%) \\
  \hline
  Noras dalyvauti & Procentai & < 80\% \\
  \hline
  Moksleivių įvertinimas & Dešimtbalė sistema & < 7 \\
  \hline
  Stovyklos kaina & Litais dalyviui & > 2000Lt \\
  \hline
  Pažeidimų skaičius & Vienetais & > 0 \\
  \hline
  Klaidų skaičius & Vienetais & \\
  \hline
  Neištaisytų klaidų skaičius & Vienetais & > 0 \\
  \hline
  Pažeidimų ir klaidų kaina & Litais & > 1000Lt \\
  \hline
  Aptarnautų moksleivių skaičius & Vienetais & < 80 \\
  \hline
\end{tabular}

\subsection{Problemos, grėsmės ir neišnaudotos galimybės}

Problemos:
\begin{itemize}
  \item Didelis paliktų neištaisytų klaidų skaičius.
\end{itemize}

Grėsmės:
\begin{itemize}
  \item Paramos gaunamos iš rėmėjų mažėjimas. 
                                        % Gali sukelti tiek populiarumo 
                                        % mažėjimas, tiek didelis klaidų
                                        % kiekis.
  %\item Padidėjus stovyklos kainai, moksleiviai gali nebevažiuoti, nes 
  % jiems bus per brangu.
  \item Moksleivių noro dalyvauti mažėjimas.
\end{itemize}

Neišnaudos galimybės:
\begin{itemize}
% TODO O kaip jei šio mato padidėjimas, taip pat leistų geriau aptarnauti
% esamus, nes būtų daugiau laiko įsiklausyti į jų specialiuosius 
% pageidavimus? Jei kriterijus būtų ne apskritai kiek moksleivių komanda
% yra pajėgi aptarnauti, bet kiek ji yra pajėgi pašalinti specialiųjų
% atvejų?
% \item Į stovykla pakviečiama mažiau moksleivių, negu gauta parama
%   suteikia galimybę pakviesti.        % TODO Ar tai tikrai yra 
                                        % neišnaudota galimybė?
% \item Į stovykla kviečiama mažiau moksleivių nei leidžia biudžetas, nes
%   organizavimo komanda nėra pajėgi jų visų aptarnauti.
  \item Nėra atsižvelgiama į visų moksleivių galimus specialiuosius 
    poreikius.
\end{itemize}

\section{Vidinė verslo proceso analizė}

Atlikus analizę paaiškėjo, jog probleminis klaidų skaičius atsiranda dėl
šių priežasčių:
\begin{enumerate}
  \item Visi darbai daromi rankomis. (Pavyzdžiui: nors ir kvietimai
    siunčiami elektroniniu paštu, bet jų tekstas yra renkamas ranka.)
  \item Netikėti vėlavimai atliekant proceso grandinės dalis. Dėl jų 
    negalima atlikti vėlesnių tai grandinei priklausančių darbų, nes
    jie yra priklausomi nuo „užstrigusio“ darbo. Dėl to paskutinę 
    savaitę susikaupia
    darbų sankaupa, su kuria nebespėjama susitvarkyti.
\end{enumerate}

Ši problema taip pat daro įtaką grėsmių atsiradimui šiais būdais:
\begin{enumerate}
    % FIXME Aprašoma ne tai, kas turėtų būti! Turėtų būti aprašoma kodėl
    % kyla grėsmė, o ne kodėl ji galėtų kilti.
  \item Padidėjus paliktų klaidų skaičiui, rėmėjai gali nuspręsti, kad 
    stovyklos organizatoriai nėra pajėgus kokybiškai panaudoti jų pinigus
    ir todėl nutraukti finansavimą, dėl ko padidėtų dalyvavimo mokestis ir
    dalis moksleivių nebegalėtų dalyvauti.
  \item Kritus stovyklos organizavimo kokybei, gali sumažėti stovyklos
    populiarumas.
\end{enumerate}

% Atlikus analizę paaiškėjo, jog tam, kad proceso vykdymo metu yra
% padaroma daug klaidų, bei tam, kad jos nėra laiku ištaisomos turi
% įtakos tiek problemoms, tiek gr:


\section{Verslo tobulinimo strategija}

Pagrindinis verslo tobulinimo siekis yra padidinti darbo efektyvumą ir 
sumažinti klaidų skaičių.

\section{Strateginiai ir operaciniai tikslai verslo tobulinimo %
  strategijai įgyvendinti} \label{section_strat_oper_tiksl}

Strateginiai tikslai:
\begin{enumerate}
  \item \label{tiksl_efek} Padidinti darbo efektyvumą.
    \begin{enumerate}
      \item \label{tiksl_auto} Įdiegti priemones leidžiančias automatizuoti 
        mechaninius darbus:
        \begin{enumerate}
          \item \label{tiksl_el} kvietimų moksleiviams siuntimas;
          \item \label{tiksl_r} moksleivių informacijos surinkimas.
        \end{enumerate}
    \end{enumerate}
  \item \label{tiksl_kl} Sumažinti klaidų skaičių.
    \begin{enumerate}
      \item \label{tiksl_vald} Įdiegti priemones leidžiančias efektyviai 
        sekti ir valdyti darbo eigą:
        \begin{enumerate}
          \item \label{tiksl_dvisk} vienoje vietoje matyti visų darbų 
            sąrašą, bei kaip jų vykdymas atitinka numatytąjį planą;
          \item \label{tiksl_dbus} peržiūrėti darbo būseną (negalimas 
            pradėti vykdyti, nes
            nebaigti kiti darbai nuo kurių šis priklauso; galimas pradėti
            vykdyti; pradėtas vykdyti; baigtas vykdyti; patikrintas);
          \item \label{tiksl_dinfo} peržiūrėti darbo informaciją (ką 
            konkrečiai reikia 
            atlikti; kokius darbus reikia atlikti prieš tai; kada 
            vėliausiai darbas turi būti pradėtas, kad būtų baigtas laiku;
            kiek vidutiniškai jį užtrunka atlikti; kokios dažniausiai 
            iškyla problemos jį vykdant; jei vėluoja, tai kiek);
          \item \label{tiksl_dvyk} peržiūrėti vykdytojus (kas vykdė/vykdo 
            šį darbą; kas atliko peržiūrą);
          \item \label{tiksl_dv} kiekvienas vykdytojas (jei jam suteiktos 
            tokios teisės) gali pasiimti darbą vykdymui arba peržiūrai.
                                        % XXX Rašant naudojimo scenarijų, 
                                        % nepamiršti patikrinti ar aktorius
                                        % turi reikiamas teises atlikti 
                                        % veiksmą. (Ne visiems leidžiama
                                        % dirbti su asmeniniais moksleivių
                                        % duomenimis.)
        \end{enumerate}
    \end{enumerate}
\end{enumerate}
