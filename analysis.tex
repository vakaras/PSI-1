\chapter{Verslo proceso analizė}

% TODO

\section{Verslo proceso aprašas}

Nuotolinio mokymo mokyklos padalinys, organizuojantis stovyklas pažangiems
moksleiviams, yra atsakingas už vidutiniškai dešimties dienų trukmės
stovyklos vidutiniškai šimtui moksleivių suorganizavimą. Šis padalinys
rūpinasi nurodytų, kaip pažangių, moksleivių, bei jų dėstytojų pakvietimu, 
apgyvendinimu, bei maitinimu. Taip pat jis atsakingas už stovyklos metu 
skaitomų paskaitų kokybę, bei moksleivių aprūpinimą visomis reikalingomis
darbui priemonėmis.

\section{Išorinė verslo proceso analizė}

\subsection{Analizės rezultatai}

Stovyklų organizavimo tikslas yra suteikti moksleiviams galimybę pagilinti
savo žinias ir kitaip patobulėti. Taigi pagrindinė proceso įeiga ir išeiga
yra tie patys moksleiviai, tik patobulėję (arba ne, jei stovykla nebuvo
sėkminga). Kadangi „patobulėjimo dydis“ yra labai santykinis dalykas ir yra 
individualus kiekvienam moksleiviui, tai jį tiesiogiai įvertinti praktiškai
yra neįmanoma. Dėl šio priežasties buvo sugalvoti kiti kriterijai, kuriais
galima vertinti stovyklos organizavimo sėkmingumą:

\begin{enumerate}
  \item motyvacijos mokytis pokytis (motyvacija mokytis) – kiek padidėjo
    ar sumažėjo moksleivio teisingai išspręstų ir visų jam duotų užduočių
    santykis;
  \item moksleivių noras dalyvauti stovykloje (noras dalyvauti) – kokia 
    dalis moksleivių, gavusių kvietimą, pareiškia norą dalyvauti ir kokie 
    yra pareiškusiųjų pasiekimai;
  \item moksleivių atsiliepimai apie stovyklą anketose;
  \item stovyklos organizavimo sąnaudos (kaina);
  \item pažeidimų (įvairiausiems teisės aktams) ir klaidų (pavyzdžiui
    rėmėjo pavadinimas kvietime atspausdintas su klaida) sukeltų padarinių
    vertė – kiek kainuotų juos pašalinti, jei būtų bandoma tai daryti + 
    moralinė žala.
\end{enumerate}

Kriterijų detalizavimas, ir jų kritinės vertės:

\begin{tabular}[]{| l | l | c |}
  \hline
  Kriterijus & Matas & Kritinė vertė \\
  \hline
  Motyvacija mokytis & Procentai & < -5\% (sumažėjo 5\%) \\
  \hline
  Noras dalyvauti & Procentai & < 80\% \\
  \hline
  Moksleivių atsiliepimai & Dešimtbalė sistema & < 7 \\
  \hline
  Kaina & Litais dalyviui & > 2000Lt \\
  \hline
  Pažeidimai ir klaidos & Litais & > 1000Lt \\
  \hline
\end{tabular}

\subsection{Problemos, grėsmės ir neišnaudotos galimybės}

* Proceso efektyvumo matai, bei kritinės jų vertės.
* Juridiniai aktai reguliuojantys verslo procesą, bei jų pateikiami 
  kriterijai?
* Įvaizdžio vertinimas. (Moksleivių pasitenkinimo anketos, kas dar?)


%Kriterijus: mokinių motyvacija daryti nuotolinio užduotis.
!Kriterijus: ne padidinti galimybes, o numušti kainą. (Sumažinti 
pasiruošimo trukmę ir sumažinti klaidų skaičių.)

TODO: Gauti sąrašą juridinių aktų, reglamentuojančių stovyklų organizavimą.

Vienintelis liečiantis juridinis dalykas – asmens duomenų privatumas.

Problemos:

* Stovyklos organizavimo darbai nebūna baigti iki jos pradžios.


\section{Vidinė verslo proceso analizė}

\section{Verslo tobulinimo strategija}

Siekis: ne padidinti galimybes, o numušti kainą. (Sumažinti 
pasiruošimo trukmę ir sumažinti klaidų skaičių.)


\section{Strateginiai ir operaciniai tikslai verslo tobulinimo %
  strategijai įgyvendinti}

Strateginiai tikslai:

* Sumažinti trukmę.

  * Išskirstyti darbus.
  * Automatizuoti kai kuriuos darbus.

* Sumažinti klaidų skaičių.

  * Automatizuoti kai kuriuos darbus.
