\chapter{Verslo proceso analizė}

\section{Verslo proceso aprašas}

Nuotolinio mokymo mokyklos padalinys, organizuojantis stovyklas pažangiems
moksleiviams, yra atsakingas už vidutiniškai dešimties dienų trukmės
stovyklos vidutiniškai šimtui moksleivių suorganizavimą. Šis padalinys
rūpinasi nurodytų, kaip pažangių, moksleivių, bei jų dėstytojų pakvietimu, 
apgyvendinimu, bei maitinimu. Taip pat jis atsakingas už stovyklos metu 
skaitomų paskaitų kokybę, bei moksleivių aprūpinimą visomis reikalingomis
darbui priemonėmis. Didžiąją dalį stovyklos organizavimui reikalingų lėšų
duoda nuotolinio mokymo mokyklos rėmėjai, o trūkumą padengia į stovyklą
važiuojantys moksleiviai, sumokėdami dalyvio mokestį. Su moksleiviais ir
dėstytojais dažniausiai bendraujama elektroniniais individualiais 
laiškais, o duomenys saugomi popieriuje, bei MS Excel skaičiatlentėse.

\section{Išorinė verslo proceso analizė}

\subsection{Analizės rezultatai}

Stovyklų organizavimo tikslas yra suteikti moksleiviams galimybę pagilinti
savo žinias ir kitaip patobulėti. Taigi pagrindinė proceso įeiga ir išeiga
yra tie patys moksleiviai. Apie į stovyklą kviečiamus moksleivius, mes
žinome, jų nuotolinio mokymosi rezultatus, bei kokia jų dalis priima 
kvietimą dalyvauti stovykloje (įeigos kriterijai). Po stovyklos
praėjus laiko tarpui, mes vėl galime įvertinti moksleivių nuotolinio
mokymosi rezultatus, be to moksleiviai užpildo įvertinimo anketas, kuriose
nurodo, kaip jiems patiko konkrečios paskaitos ir visa stovykla apskritai
(išeigos kriterijai). Taip pat organizuojant stovyklą yra svarbu jos 
organizavimo kaina (nes jai didėjant dalyvio mokestis kai kuriems 
moksleiviams gali tapti per dideliu), bei organizuojant padarytų klaidų
(nusižengimai įstatymams, susitarimams, bei darbo klaidos) kiekis.

Apibendrindami, galime išskirti tokius stovyklų organizavimo proceso 
vertinimo kriterijus:
\begin{enumerate}
  \item motyvacijos mokytis pokytis (motyvacija mokytis) – kiek padidėjo
    ar sumažėjo moksleivio teisingai išspręstų ir visų jam duotų užduočių
    santykis;
  \item moksleivių noras dalyvauti stovykloje (noras dalyvauti) – kokia 
    dalis moksleivių, gavusių kvietimą, pareiškia norą dalyvauti;
  \item moksleivių atsiliepimai apie stovyklą pasitenkinimo anketoje;
  \item stovyklos organizavimo sąnaudos (kaina);
  \item juridinių aktų (įstatymų (buhalterinės apskaitos,
    mokesčių administravimo, asmens duomenų teisinės apsaugos
    ir kiti), bei susitarimų) pažeidimų skaičius;
  \item padarytų darbinių klaidų (pavyzdžiui rėmėjo pavadinimas 
    atspausdintas su klaida, arba kvietimas išsiųstas ne tam moksleiviui, 
    kuriam reikėjo) skaičius;
  \item neištaisytų klaidų skaičius;
  \item pažeidimų  ir klaidų sukeltų padarinių vertė – kiek kainuotų juos 
    pašalinti, jei būtų bandoma tai daryti + moralinė žala.
\end{enumerate}

Kriterijų detalizavimas, ir jų kritinės vertės:

\begin{tabular}[]{| l | p{2.2cm} | c |}
  \hline
  Kriterijus & Matas & Kritinė vertė \\
  \hline
  Motyvacija mokytis & Procentai & < -5\% (sumažėjo 5\%) \\
  \hline
  Noras dalyvauti & Procentai & < 80\% \\
  \hline
  Moksleivių įvertinimas & Dešimtbalė sistema & < 7 \\
  \hline
  Stovyklos kaina & Litais dalyviui & > 2000Lt \\
  \hline
  Pažeidimų skaičius & Vienetais & > 0 \\
  \hline
  Klaidų skaičius & Vienetais & \\
  \hline
  Neištaisytų klaidų skaičius & Vienetais & > 0 \\
  \hline
  Pažeidimų ir klaidų kaina & Litais & > 1000Lt \\
  \hline
\end{tabular}

\subsection{Problemos, grėsmės ir neišnaudotos galimybės}

Problemos:
\begin{itemize}
  \item Didelis paliktų neištaisytų klaidų skaičius.
\end{itemize}

Grėsmės:
\begin{itemize}
  \item Padidėjus stovyklos kainai, moksleiviai gali nebevažiuoti, nes 
    jiems bus per brangu.
  \item Moksleivių noro dalyvauti mažėjimas.
  \item Paramos gaunamos iš rėmėjų mažėjimas. 
                                        % Gali sukelti tiek populiarumo 
                                        % mažėjimas, tiek didelis klaidų
                                        % kiekis.
\end{itemize}

% Neišnaudotos galimybės TODO – sugalvoti.

\section{Vidinė verslo proceso analizė}


\section{Vidinė verslo proceso analizė}

\section{Verslo tobulinimo strategija}

Siekis: ne padidinti galimybes, o numušti kainą. (Sumažinti 
pasiruošimo trukmę ir sumažinti klaidų skaičių.)


\section{Strateginiai ir operaciniai tikslai verslo tobulinimo %
  strategijai įgyvendinti}

Strateginiai tikslai:

* Sumažinti trukmę.

  * Išskirstyti darbus.
  * Automatizuoti kai kuriuos darbus.

* Sumažinti klaidų skaičių.

  * Automatizuoti kai kuriuos darbus.
