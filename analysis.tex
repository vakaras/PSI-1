\chapter{Verslo proceso analizė}

\section{Verslo proceso aprašas}

Nuotolinio mokymo mokyklos padalinys, organizuojantis stovyklas pažangiems
moksleiviams, yra atsakingas už vidutiniškai dešimties dienų trukmės
stovyklos vidutiniškai šimtui moksleivių suorganizavimą. Šis padalinys
rūpinasi nurodytų, kaip pažangių, moksleivių, bei jų dėstytojų pakvietimu, 
apgyvendinimu, bei maitinimu. Taip pat jis atsakingas už stovyklos metu 
skaitomų paskaitų kokybę, bei moksleivių aprūpinimą visomis reikalingomis
darbui priemonėmis. Didžiąją dalį stovyklos organizavimui reikalingų lėšų
duoda nuotolinio mokymo mokyklos rėmėjai, o trūkumą padengia į stovyklą
važiuojantys moksleiviai, sumokėdami dalyvio mokestį. Su moksleiviais ir
dėstytojais dažniausiai bendraujama individualiais elektroniniais 
laiškais, o duomenys saugomi popieriuje, bei MS Excel skaičiuoklės knygose.

\section{Išorinė verslo proceso analizė}

\subsection{Analizės rezultatai}

Stovyklų organizavimo tikslas yra suteikti moksleiviams galimybę pagilinti
savo žinias ir kitaip patobulėti. Taigi pagrindinė proceso įeiga ir išeiga
yra tie patys moksleiviai. Apie į stovyklą kviečiamus moksleivius, yra
žinomi jų nuotolinio mokymosi rezultatai, bei kokia jų dalis priima 
kvietimą dalyvauti stovykloje (įeigos kriterijai). Po stovyklos
praėjus laiko tarpui, vėl galima įvertinti moksleivių nuotolinio
mokymosi rezultatus. Be to moksleiviai užpildo įvertinimo anketas, kuriose
nurodo, kaip jiems patiko konkrečios paskaitos ir visa stovykla
(išeigos kriterijai). Taip pat organizuojant stovyklą yra svarbu jos 
organizavimo kaina (nes jai didėjant dalyvio mokestis kai kuriems 
moksleiviams gali tapti per didelis), bei organizuojant padarytų klaidų
(nusižengimai įstatymams, susitarimams, bei darbo klaidos) kiekis. % TODO :paaiškinti „darbo klaidos“

Apibendrinant galima išskirti tokius stovyklų organizavimo proceso 
vertinimo kriterijus:
\begin{enumerate}% TODO :perdaryti, pridėti refference
  \item motyvacijos mokytis pokytis – kiek padidėjo
    ar sumažėjo moksleivio teisingai išspręstų ir visų jam duotų užduočių
    santykis; 
  \item moksleivių noras dalyvauti stovykloje – kokia 
    dalis moksleivių, gavusių kvietimą, pareiškia norą dalyvauti;
  \item moksleivių atsiliepimai apie stovyklą pasitenkinimo anketoje;
  \item stovyklos organizavimo sąnaudos;
  \item juridinių aktų (įstatymų (buhalterinės apskaitos,
    mokesčių administravimo, asmens duomenų teisinės apsaugos
    ir kiti) bei susitarimų) pažeidimų skaičius;
  \item padarytų darbinių klaidų skaičius; % FIXME : sukurti nuorodą į gloss
  \item neištaisytų klaidų skaičius;
  \item pažeidimų  ir klaidų sukeltų padarinių vertė – kiek kainuotų juos 
    pašalinti, jei būtų bandoma tai daryti, bei moralinė žala;

\end{enumerate}

Kriterijų detalizavimas ir jų kritinės vertės:

% TODO : Stulpelyje kriterijus žodžius pakeisti nuorodomis

% \begin{tabular}[]{| l | p{2.2cm} | c |}
%   \hline
%   Kriterijus & Matas & Kritinė vertė \\
%   \hline
%   Motyvacija mokytis & Procentai & < -5\% (sumažėjo 5\%) \\
%   \hline
%   Noras dalyvauti & Procentai & < 80\% \\
%   \hline
%   Moksleivių įvertinimas & Dešimtbalė sistema & < 7 \\
%   \hline
%   Stovyklos kaina & Litais dalyviui & > 2000Lt \\
%   \hline
%   Pažeidimų skaičius & Vienetais & > 0 \\
%   \hline
%   Klaidų skaičius & Vienetais & $\infty$\\
%   \hline
%   Neištaisytų klaidų skaičius & Vienetais & > 0 \\ 
%   \hline
%   Pažeidimų ir klaidų kaina & Litais & > 1000Lt \\
%   \hline
% \end{tabular}

\subsection{Problemos, grėsmės ir neišnaudotos galimybės}

Problemos:
\begin{itemize}
  \item Didelis paliktų neištaisytų klaidų skaičius.
\end{itemize}

Grėsmės:
\begin{itemize}
  \item Paramos gaunamos iš rėmėjų mažėjimas. 
                                        % Gali sukelti tiek populiarumo 
                                        % mažėjimas, tiek didelis klaidų
                                        % kiekis.
  %\item Padidėjus stovyklos kainai, moksleiviai gali nebevažiuoti, nes 
  % jiems bus per brangu.
  \item Moksleivių noro dalyvauti mažėjimas.
\end{itemize}

Neišnaudos galimybės:
\begin{itemize}
% TODO O kaip jei šio mato padidėjimas, taip pat leistų geriau aptarnauti
% esamus, nes būtų daugiau laiko įsiklausyti į jų specialiuosius 
% pageidavimus? Jei kriterijus būtų ne apskritai kiek moksleivių komanda
% yra pajėgi aptarnauti, bet kiek ji yra pajėgi pašalinti specialiųjų
% atvejų?
% \item Į stovykla pakviečiama mažiau moksleivių, negu gauta parama
%   suteikia galimybę pakviesti.        % TODO Ar tai tikrai yra 
                                        % neišnaudota galimybė?
% \item Į stovykla kviečiama mažiau moksleivių nei leidžia biudžetas, nes
%   organizavimo komanda nėra pajėgi jų visų aptarnauti.
  \item Nėra atsižvelgiama į visų moksleivių galimus specialiuosius 
    poreikius.
                  % TODO :Paaiškinti spec.poreikius
\end{itemize}

\section{Vidinė verslo proceso analizė}

Atlikus analizę paaiškėjo, jog probleminis klaidų skaičius atsiranda dėl
šių priežasčių:
\begin{enumerate}
  \item visi darbai daromi rankomis (pavyzdžiui: nors ir kvietimai
    siunčiami elektroniniu paštu, bet jų tekstas yra renkamas ranka);
  \item netikėti vėlavimai atliekant proceso grandinės dalis. Dėl jų 
    negalima atlikti vėlesnių tai grandinei priklausančių darbų, nes
    jie yra priklausomi nuo „užstrigusio“ darbo. Dėl to paskutinę 
    savaitę susikaupia
    darbų sankaupa, su kuria nebespėjama susitvarkyti.
\end{enumerate}

Ši problema taip pat daro įtaką grėsmių atsiradimui šiais būdais:
\begin{enumerate}
  \item padidėjus paliktų klaidų skaičiui rėmėjai nusprendžia, kad 
    stovyklos organizatoriai nėra pajėgūs kokybiškai panaudoti jų paramą
    ir todėl sumažina finansavimą. Dėl to padidėja dalyvio mokestis ir
    dalis moksleivių nebegali dalyvauti;
  \item kritus stovyklos organizavimo kokybei gali sumažėja stovyklos
    populiarumas.
\end{enumerate}

\section{Verslo tobulinimo strategija}

Pagrindinis verslo tobulinimo siekis yra padidinti darbo efektyvumą ir 
sumažinti klaidų skaičių.

\section{Strateginiai ir operaciniai tikslai verslo tobulinimo %
  strategijai įgyvendinti} \label{section_strat_oper_tiksl}

\begin{easylist}
& \label{tiksl_efek} 
  Padidinti darbo efektyvumą.
&&  \label{tiksl_auto} 
    Įdiegti priemones leidžiančias automatizuoti mechaninius darbus:
&&& \label{tiksl_el} 
      kvietimų moksleiviams siuntimas;
&&& \label{tiksl_r} 
      informacijos iš moksleivių surinkimas.
& \label{tiksl_kl} 
  Sumažinti klaidų skaičių.
&& \label{tiksl_vald} 
    Įdiegti priemones leidžiančias efektyviai sekti ir valdyti darbo eigą:
&&& \label{tiksl_dvisk} 
      vienoje vietoje matyti visų darbų sąrašą, bei kaip jų vykdymas 
      atitinka numatytąjį planą;
&&& \label{tiksl_dbus} 
      peržiūrėti darbo būseną (negalimas pradėti vykdyti, nes nebaigti 
      kiti darbai nuo kurių šis priklauso; galimas pradėti vykdyti; 
      pradėtas vykdyti; baigtas vykdyti; patikrintas);
&&& \label{tiksl_dinfo} 
      peržiūrėti darbo informaciją (ką konkrečiai reikia atlikti; kokius 
      darbus reikia atlikti prieš tai; kada vėliausiai darbas turi būti 
      pradėtas, kad būtų baigtas laiku; kiek vidutiniškai jį užtrunka 
      atlikti; kokios dažniausiai iškyla problemos jį vykdant; jei vėluoja, 
      tai kiek);
&&& \label{tiksl_dvyk} 
      peržiūrėti vykdytojus (kas vykdė/vykdo šį darbą; kas atliko peržiūrą);
&&& \label{tiksl_dv} 
      kiekvienas vykdytojas (jei jam suteiktos tokios teisės) gali pasiimti 
      darbą vykdymui;
&&& \label{tiksl_vad}
      vadovas gali patikrinti, ar gerai atliktas darbas ir 
      patvirtinti/atmesti jo atlikimą.
\end{easylist}
