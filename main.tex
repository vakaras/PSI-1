\documentclass{report}

\usepackage{fontspec}
\usepackage{xltxtra}
\usepackage[lithuanian]{babel}
\usepackage{indentfirst}
\usepackage[]{amsmath}
\usepackage{amsthm}
\usepackage{amsfonts}
\usepackage{alltt}
\usepackage[]{hyperref}
\usepackage[all]{hypcap}
\usepackage[toc,acronym,xindy,translate=false]{glossaries}
\usepackage{glossaries-babel}

\hypersetup{pdfborder={0 0 0 0}}

\defaultfontfeatures{Mapping=tex-text}

% Numeravimo stiliaus pakeitimas.
\renewcommand{\theenumii}{\arabic{enumii}}
\renewcommand{\labelenumii}{\theenumii}
\renewcommand{\theenumiii}{\arabic{enumiii}}
\renewcommand{\labelenumiii}{\theenumiii}

% Nurodymas sugeneruoti terminų žodynėlį.
\makeglossaries

% TODO Susitvarkyti su numeravimo atvaizdavimu. (nested enumerate; 
% label-ref)

\addto\captionslithuanian{%
  \renewcommand*{\glossaryname}{Terminų žodynėlis}}
              % LaTeX raktažodžių vertimai.
\newglossaryentry{siekis}{
  name=siekis,
  description={
    Siekiu vadinamas pagrindinis tikslas, kurio numatoma siekti naudojant 
    sukurtąją sistemą. Siekiu apibūdinama nauda, kurios galima tikėtis iš 
    sistemos, ir pagrindžiamas tos sistemos kūrimo tikslingumas
    %\cite[100]{cap_psi1} FIXME Išsiaiškinti kodėl neveikia.
    },
  sort=siekis,
  plural=siekiai
  }

\newglossaryentry{strateginis-tikslas}{
  name={strateginis tikslas},
  description={
    Strateginiai sistemos naudojimo tikslai – tai pagrindinio jos naudojimo
    tikslo potiksliai, nusakantys pačius bendriausius, ilgalaikius
    kuriamos sistemos užsakovo ketinimus
    %\cite[100]{cap_psi1} FIXME Išsiaiškinti kodėl neveikia.
    },
  sort={strateginis tikslas},
  plural={strateginiai tikslai}
  }

\newglossaryentry{operacinis-tikslas}{
  name={operacinis tikslas},
  description={
    Operaciniai sistemos naudojimo tikslai (sistemos sprendžiami 
    uždaviniai) – tai konstruktyvūs, matuojami ir terminuoti strateginių
    tikslų potiksliai
    %\cite[100]{cap_psi1} FIXME Išsiaiškinti kodėl neveikia.
    },
  sort={operacinis tikslas},
  plural={operaciniai tikslai}
  }

\newglossaryentry{saitynas}{
  name={saitynas},
  description={
    Hipertekstinės informacijos visuotinis tinklas, svarbiausias interneto
    komponentas
    },
  sort=saitynas
  }

\newglossaryentry{duomenų-bazė}{
  name={duomenų bazė},
  description={
    Duomenų rinkinys, susistemintas ir sutvarkytas taip, kad juo būtų
    galima patogiai naudotis
    },
  sort={duomenų bazė},
  plural={duomenų bazės}
  }
                  % Terminų žodynėlis.

\begin{document}

\begin{titlepage}

  \begin{center}

    % Puslapio viršus.
    \textsc{\Large Vilniaus universitetas\\
    Matematikos ir informatikos fakultetas\\
    Programų sistemų katedra}\\[6.0cm]

    % Projekto pavadinimas.
    \textbf{ \LARGE „Nuotolinio mokymo mokyklos padalinio, 
    organizuojančio stovyklas pažangiems moksleiviams, pagalbinė darbų 
    valdymo ir skirstymo sistema“ }\\
    { \Large (Skirstytuvas)}\\[0.5cm]

    {\Large Verslo tikslų ir poreikių specifikacija }\\[0.5cm]

    {\Large (1.0rc versija)}\\[3.0cm]

    \begin{minipage}[]{0.8\textwidth}
      \begin{flushright} 
        Darbą atliko 2 kurso 2 grupės studentai: \\
        Vytautas Astrauskas \\
        Vytautas Butkus \\
        Egidijus Lukauskas \\
        Ernestas Monginas
      \end{flushright}
    \end{minipage}

    \vfill

    \textbf{\large  Vilnius \\ \the\year }
  \end{center}
  
\end{titlepage}

\begin{abstract}
  V. Astrauskas, V. Butkus, E. Lukauskas, E. Monginas. Nuotolinio 
  mokymo mokyklos padalinio, organizuojančio stovyklas pažangiems
  mokiniams, tikslų ir poreikių specifikacija (% TODO
  1 versija). VU MIF PS katedra, Vilnius 2010.

  Šiame darbe pateiktas kurso „Programų sistemų inžinerija“ laboratorinis
  darbas, skirtas verslo tikslų ir poreikių specifikavimui. Tai 
  pirmasis iš keturių pagal šį kursą daromų laboratorinių darbų.
  Darbas skirtas įsigilinti į užsakovo verslą, nustatyti problemas,
  grėsmes bei neišnaudotas galimybes ir nuspręsti ar įmanoma būtų
  sukurti programų sistemą, kuri padėtų spręsti šias problemas bei
  įgyvendinti neišnaudotas galimybes. Jei programų sistema gali
  pagerinti situaciją, nustatoma kokias būtent problemas ji padės spręsti
  bei kokias neišnaudotas galimybes ji padės įgyvendinti. Darbe atlikta
  verslo proceso analizė, pasiūlyta šio proceso tobulinimo strategija
  ir nustatyta kokias paslaugas turėtų teikti šią strategiją palaikanti 
  programų sistema. Atlikta sistemos įgyvendinamumo analizė bei nustatyta
  kokią konkrečią naudą ji teiks naudotojams.
  
  Informacija apie vykdytojus ir jų įnašą į darbą:

  \begin{itemize}
    \item Vytautas Astrauskas (\url{Vytautas.Astrauskas@mif.stud.vu.lt}): 
      \begin{itemize}
        \item Verslo proceso aprašas.
        \item Išorinė verslo proceso analizė.
        \item Vidinė verslo proceso analizė.
        \item Verslo tobulinimo strategija.
        \item Strateginiai ir operaciniai tikslai verslo tobulinimo
          strategijai įgyvendinti.
        \item Užsakovo poreikių analizė.
        \item Esamoji būklė.
      \end{itemize}

    \item Vytautas Butkus (\url{Vytautas.Butkus.2@mif.stud.vu.lt}): 
      \begin{itemize}
        \item Juridinis įgyvendinamumas.
        \item Sistemos panaudojimas.
        \item Sistemos teikiama nauda.
      \end{itemize}

    \item Egidijus Lukauskas (\url{Egidijus.Lukauskas@mif.stud.vu.lt}): 
      \begin{itemize}
        \item Scenarijaus aprašas.
        \item Priemonės scenarijui įgyvendinti.
      \end{itemize}

    \item Ernestas Monginas (\url{Ernestas.Monginas@mif.stud.vu.lt}):
      \begin{itemize}
        \item Operacinis įgyvendinamumas.
        \item Techninis įgyvendinamumas.
        \item Ekonominis įgyvendinamumas.
      \end{itemize}

  \end{itemize}
  
\end{abstract}


\tableofcontents

\chapter{Įvadas}

% TODO: žr. 5 reikalavimų psl.

\section{Programų sistemos pavadinimas}
% TODO

\section{Dalykinė sritis}
% TODO

\section{Probleminė sritis}
% TODO

\section{Naudotojai}
% TODO

\section{Darbo pagrindas}
% TODO

%TODO: ištrinti, kai bus normalių citatų.
Beprasmė citata iš Čaplinsko vadovėlio: „Objektinio stiliaus architektūros
grindžiamos objektine paradigma. Ji šiuo metu yra 
vyraujanti.“\cite[50]{cap_psi2}

\section{Naudoti dokumentai}
% TODO

\bibliographystyle{plain}
\bibliography{bibliography}

\chapter{Verslo proceso analizė}

% TODO

\section{Verslo proceso aprašas}

Nuotolinio mokymo mokyklos padalinys, organizuojantis stovyklas pažangiems
moksleiviams, yra atsakingas už vidutiniškai dešimties dienų trukmės
stovyklos vidutiniškai šimtui moksleivių suorganizavimą. Šis padalinys
rūpinasi nurodytų, kaip pažangių, moksleivių, bei jų dėstytojų pakvietimu, 
apgyvendinimu, bei maitinimu. Taip pat jis atsakingas už stovyklos metu 
skaitomų paskaitų kokybę, bei moksleivių aprūpinimą visomis reikalingomis
darbui priemonėmis.

\section{Išorinė verslo proceso analizė}

* Proceso efektyvumo matai, bei kritinės jų vertės.
* Juridiniai aktai reguliuojantys verslo procesą, bei jų pateikiami 
  kriterijai?
* Įvaizdžio vertinimas. (Moksleivių pasitenkinimo anketos, kas dar?)

%Kriterijus: mokinių motyvacija daryti nuotolinio užduotis.
!Kriterijus: ne padidinti galimybes, o numušti kainą. (Sumažinti 
pasiruošimo trukmę ir sumažinti klaidų skaičių.)

TODO: Gauti sąrašą juridinių aktų, reglamentuojančių stovyklų organizavimą.

Vienintelis liečiantis juridinis dalykas – asmens duomenų privatumas.

Problemos:

* Stovyklos organizavimo darbai nebūna baigti iki jos pradžios.


\section{Vidinė verslo proceso analizė}

\section{Verslo tobulinimo strategija}

Siekis: ne padidinti galimybes, o numušti kainą. (Sumažinti 
pasiruošimo trukmę ir sumažinti klaidų skaičių.)


\section{Strateginiai ir operaciniai tikslai verslo tobulinimo %
  strategijai įgyvendinti}

Strateginiai tikslai:

* Sumažinti trukmę.

  * Išskirstyti darbus.
  * Automatizuoti kai kuriuos darbus.

* Sumažinti klaidų skaičių.

  * Automatizuoti kai kuriuos darbus.

\chapter{Užsakovo poreikių analizė}

% TODO Kaip daryti lenteles su LaTeX parašyta čia:
% http://en.wikibooks.org/wiki/LaTeX/Tables

\begin{tabular}[]{| l | p{1.6cm} | p{5.8cm} | c |}
  \hline
  Nr. & Operacinis tikslas & Resursai, reikalingi tikslui įgyvendinti &
  Prioritetas \\
  \hline
  1. & \ref{tiksl_el} & TODO & ? \\
  \hline
\end{tabular}


\chapter{Sistemos naudojimo scenarijus}

\section{Esamoji būklė}

Mokyklos būstinėje yra vienas nešiojamas kompiuteris su Windows operacine
sistema, kuriame laikomi visi „einamieji“ duomenys, bei prie jo prijungtas 
lazerinis spausdintuvas. Nuotolinio mokymo mokykla taip pat turi savo 
svetainę, bei duomenų bazę su informacija apie moksleivius ir dėstytojus.
Darbuotojai dirba su savo asmeniniais kompiuteriais, kuriuose naudoja
įvairią programinę įrangą (pavyzdžiui naudojamų operacinių sistemų 
sąraše yra „Windows XP“, „Windows 7“, „Ubuntu Linux“, „Xubuntu Linux“,
„Gentoo Linux“). Dauguma darbuotojų turi didelę darbo su biuro 
programomis patirtį.
% TODO Išversti ir papildyti.

\section{Scenarijaus aprašas}

\ref{fig:uml_usecase} diagramoje pavaizduota darbo su sistema UML schema.



\begin{figure}[htb]
  \begin{center}
    \includegraphics[scale=0.7]{images/Seka.png}
    \caption{UML schema vaizduojanti darbą su įdiegta sistema.}
  \end{center}
  \label{fig:uml_usecase}
\end{figure}


\subsection{Darbo vietų aprašas}
\begin{enumerate}
  \item Vadovas
	\begin{itemize}
	  \item Techninė įranga
		\begin{enumerate}
			\item Kompiuteris su internetu
		\end{enumerate}
	  \item Programinė įranga
		\begin{enumerate}
			\item Operacinė sistema, interneto naršyklė
		\end{enumerate}
	  \item Kvalifikaciniai reikalavimai
		\begin{enumerate}
			\item Kompiuterinio raštingumo pagrindai
			\item Sistemos "Skirstytuvas" administavimo apmokymas % FIXME
		\end{enumerate}
	\end{itemize}

  \item Vykdytojas
	\begin{itemize}
	  \item Techninė įranga
		\begin{enumerate}
			\item Kompiuteris su internetu
		\end{enumerate}
	  \item Programinė įranga
		\begin{enumerate}
			\item Operacinė sistema, interneto naršyklė
		\end{enumerate}
	  \item Kvalifikaciniai reikalavimai
		\begin{enumerate}
			\item Kompiuterinio raštingumo pagrindai
			\item Sistemos "Skirstytuvas" naudojimo apmokymas
		\end{enumerate}
	\end{itemize}
\end{enumerate}

\section{Priemonės scenarijui įgyvendinti}
\begin{enumerate}
	\item Virtuali tarnybinė stotis
	\item Tinklo įranga
	\item Pašto serveris
	\item Duomenų bazių valdymo sistemos
	\item Interneto ryšio paslaugos
	\item Operacinės sistemos
	\item Darbuotojų apmokymas naudotis sistema
	\item Interneto svetainės vardo sritis .lt zonoje
\end{enumerate}
\chapter{Sistemos \k{i}gyvendinamumo ir teikiamos naudos analiz\.{e}}

% TODO


\section{Operacinis \k{i}gyvendinamumas}

\begin{tabular}{|c|c|}
\hline 
Inovacinis slenkstis & Inovacinio slenks\v{c}io pašalinimas\tabularnewline
\hline 
Darbuotojai neturi darbo su nauja sistema patirties & Organizuoti darbo su nauja sistema kursus\tabularnewline
\hline 
Moksleiviai nežino apie nauj\k{a} sistem\k{a} & Informuoti moksleivius apie nauj\k{a} sistem\k{a}\tabularnewline
\hline
\end{tabular}

\section{Techninis \k{i}gyvendinamumas}

Grup\.{e}, kurianti sistem\k{a}, patirties tokioje srityje dar neturi,
ta\v{c}iau yra išklausiusi vieneri\k{u} met\k{u} program\k{u} sistem\k{u}
Vilniaus Universitete kurs\k{a}. Komandos nariai susipažin\k{e} su
duomen\k{u} bazi\k{u} valdymo sistemomis, turi matematin\.{e}s logikos
pagrindus, algoritmavimo bei programavimo patirties(C, C++ , Java,
Python kalbomis), taip pat komandos nariai turi Linux serverio 
administravimo
pagrindus. Vis\k{u} nari\k{u} CV yra pridedami kaip priedai. Taigi
grup\.{e} yra visiškai paj\.{e}gi kurti tokio pob\={u}džio program\k{u}
sistem\k{a}.


\section{Ekonominis \k{i}gyvendinamumas}

Technin\.{e} \k{i}ranga:
\begin{itemize}
  \item Tinklo \k{i}ranga - 200 Lt.
\end{itemize}

Programin\.{e} \k{i}ranga:
\begin{itemize}
  \item Sistema - 10000 Lt.
\end{itemize}

Kita:
\begin{itemize}
  \item Darbuotoj\k{u} apmokymas - 1000 Lt.
\end{itemize}

Viso: 11200 Lt.

Metin\.{e}s eksplotavimo išlaidos:
\begin{itemize}
  \item Interneto ryšio paslaugos - 600 Lt;
  \item Virtuali tarnybin\.{e} stotis (operacin\.{e} sistema, pašto 
    serveris, duomen\k{u} bazi\k{u} valdymo sistema) - 720 Lt;
  \item Vardin\.{e}s srities mokestis - 31 Lt.
\end{itemize}

Viso: 1351 Lt.

\section{Juridinis įgyvendinamumas}

Kuriama sistema neprieštarauja Lietuvos Respublikos įstatymams. Stovyklos 
dalyvių asmens duomenų rinkimas neprieštarauja Lietuvos Respublikos 
Asmens duomenų teisinės apsaugos įstatymui \cite{istat_duom_apsaug}.
Kiekvienam dalyviui bus 
suteikiama galimybė nesutikti su jo asmens duomenų rinkimu arba 
peržiūrėti jau surinktus duomenis ir panorėjus juos pakeisti. 
Pasibaigus stovyklos organizavimo ir vykdymo laikotarpiui, nereikalingi 
duomenys bus sunaikinami. Taip pat numatoma laikytis visų duomenų saugumo 
kriterijų.

\section{Sistemos panaudojimas}

\ref{fig:uml_tasks} UML schemoje vaizduojamos aktorių vykdomos užduotys ir 
kaip jas padeda vykdyti sistema.

\begin{figure}[h!]
  \begin{center}
    \includegraphics[scale=0.8]{images/sistemos_panaudojimas.png}
  \end{center}
  \caption{UML schema vaizduojamos aktorių vykdomos užduotys ir kaip jas
    padeda vykdyti sistema.}
  \label{fig:uml_tasks}
\end{figure}

\section{Sistemos teikiama nauda}

\ref{fig:uml_tasks2} UML schema vaizduojamos aktorių vykdomos užduotys 
ir kaip jas
padeda vykdyti sistema.

\begin{figure}[h!]
  \begin{center}
    \includegraphics[scale=0.8]{images/sistemos_tiekiama_nauda.png}
  \end{center}
  \caption{UML schema vaizduojamos aktorių vykdomos užduotys ir kaip jas
    padeda vykdyti sistema.}
  \label{fig:uml_tasks2}
\end{figure}

\chapter{Išvados}

Sistema „Skirstytuvas“ yra skirta padidinti darbo efektyvumą ir sumažinti 
klaidų skaičių, organizuojant stovyklas pažangiems mokiniams. Taip pat 
sistema sumažina vykdytojų darbo krūvį. Tuo pačiu suteikia galimybę 
išlaikyti rėmėjus bei sumažina tikimybę prarasti mokinius dėl per
didelės stovyklos kainos. Iš operacinio įgyvendinamumo lentelės 
(\ref{tab:igyv_oper})
matome, kad inovaciniai slenksčiai gali būti nesunkiai pašalinti.
Remiantis atliktomis analizėmis nustatyta, kad sistema yra techniškai, 
ekonomiškai ir juridiškai įgyvendinama bei verta kurti.


\clearpage
\phantomsection
\addcontentsline{toc}{chapter}{Literatūra}
\bibliographystyle{plain}
\bibliography{bibliography}

\printglossary

\input{./appendixes.tex}

\end{document}
